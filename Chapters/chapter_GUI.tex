%!TEX root = ../template.tex
%%%%%%%%%%%%%%%%%%%%%%%%%%%%%%%%%%%%%%%%%%%%%%%%%%%%%%%%%%%%%%%%%%%%
%% chapter2.tex
%% NOVA thesis document file
%%
%% Chapter with the template manual
%%%%%%%%%%%%%%%%%%%%%%%%%%%%%%%%%%%%%%%%%%%%%%%%%%%%%%%%%%%%%%%%%%%%

\typeout{NT FILE chapter2.tex}

\chapter{Building the GUI}
\label{cha:users_manual}

\glsresetall

\section{Preparing the Setup} % (fold)
\label{sec:setup_prep}

For the construction of the GUI it was chosen the Visual Studio Code, a code editor SOFTWARE/FRAMEWORK developed by Microsoft, and Qt Design Studio, a Qt software designed to aid the developers to create GUI's in a WYSIWYG manner.\\




% \printbibliography[heading=subbibliography, segment=\therefsegment, title={\bibname\ for chapter~\thechapter}]




% \printbibliography[heading=subbibliography, segment=\therefsegment, title={\bibname\ for chapter~\thechapter}]
